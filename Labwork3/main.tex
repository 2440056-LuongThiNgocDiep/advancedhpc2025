\documentclass[hidelinks,12pt,a4paper]{article}

% ====== Packages ======
\usepackage{graphicx}
\usepackage{amsmath}
\usepackage{siunitx}
\usepackage{tikz}
\usetikzlibrary{automata, arrows, positioning}
\usepackage{hyperref}
\usepackage{cite}
\usepackage[T1]{fontenc}
\usepackage[utf8]{inputenc}
\usepackage{xcolor}
\usepackage{geometry}
\geometry{left=1.2in,top=1in,bottom=1in,right=1in}
\usepackage{titlesec}
\usepackage{fancyhdr}
\usepackage{float}
\usepackage{tcolorbox}
\usepackage{listings}
\usepackage{subcaption}

% ====== Color and Title Formatting ======
\definecolor{blue}{RGB}{31,56,100}

\titleformat{\section}
  {\normalfont\LARGE\bfseries}
  {\thesection}{1em}{}
\titleformat{\subsection}
  {\normalfont\Large\bfseries}
  {\thesubsection}{1em}{}
\titleformat{\subsubsection}
  {\normalfont\large\bfseries}
  {\thesubsubsection}{1em}{}

% ====== Document ======
\begin{document}
\pagenumbering{gobble}

% ====== Title Page ======
\begin{titlepage}
    \begin{tikzpicture}[remember picture,overlay,inner sep=0,outer sep=0]
        \draw[black!70!black,line width=1.5pt]
            ([xshift=-0.65in,yshift=-1cm]current page.north east) --
            ([xshift=0.65in,yshift=-1cm]current page.north west) --
            ([xshift=0.65in,yshift=1cm]current page.south west) --
            ([xshift=-0.65in,yshift=1cm]current page.south east) -- cycle;
    \end{tikzpicture}

    \begin{center}
        {\large\uppercase{University of Science and Technology of Hanoi}} \\[1.5cm]
        \includegraphics[width=0.6\linewidth]{images/usth.png} \\[1cm]
        {\huge \bfseries \uppercase{Labwork Report}} \\[1cm]
        {\large \bfseries 2440056 -- Luong Thi Ngoc Diep} \\[0.5cm]
        {\huge \bfseries {Advanced Programming for HPC}} \\[1cm]
        \rule{\linewidth}{0.3mm} \\[0.4cm]
        {\Huge \bfseries \color{blue} Labwork 3: Hello, CUDA!} \\
        \rule{\linewidth}{0.3mm} \\[0.7cm]
        \large Academic Year: 2024--2026
    \end{center}
\end{titlepage}

\clearpage
\pagenumbering{arabic}

% ====== Section 1 ======
\section{Introduction}
In this labwork, I implemented an image RGB-to-grayscale converter using both CPU and GPU with Numba CUDA.  
The goal was to compare execution time and evaluate GPU performance when using different block sizes.

% ====== Section 2 ======
\section{Implementation and Results}

\subsection{CPU Implementation}
The CPU version converts each pixel using the following formula:
\[
gray = \frac{R + G + B}{3}
\]

\begin{tcolorbox}[colback=gray!5!white, colframe=blue!40!black, title=CPU Python Code]
\begin{lstlisting}[language=Python]
for r, g, b in zip(R, G, B):
    gray_pixel = (r + g + b) / 3
    gray_img.append(gray_pixel)
\end{lstlisting}
\end{tcolorbox}

Runtime: \textbf{0.893 s} for a 604×900 image (543,600 pixels).  

\begin{figure}[H]
  \centering
  \begin{subfigure}[b]{0.45\textwidth}
    \includegraphics[width=\linewidth]{images/Original Image.png}
    \caption{Original Image}
  \end{subfigure}
  \hfill
  \begin{subfigure}[b]{0.45\textwidth}
    \includegraphics[width=\linewidth]{images/gray_cpu.png}
    \caption{Grayscale (CPU)}
  \end{subfigure}
  \caption{CPU grayscale result}
\end{figure}

\subsection{GPU Implementation}
The GPU version parallelizes the same formula using CUDA.

\begin{tcolorbox}[colback=gray!5!white, colframe=blue!40!black, title=CUDA Kernel Code]
\begin{lstlisting}[language=Python]
@cuda.jit
def grayscale(src, dst):
    i = cuda.threadIdx.x + cuda.blockIdx.x * cuda.blockDim.x
    if i < src.shape[0]:
        g = np.uint8((src[i,0] + src[i,1] + src[i,2]) / 3)
        dst[i,0] = dst[i,1] = dst[i,2] = g
\end{lstlisting}
\end{tcolorbox}

Runtime (block size 32–256): \textbf{≈0.00033 s}.  
Speedup: \textbf{~2700× faster than CPU.}

\begin{figure}[H]
  \centering
  \begin{subfigure}[b]{0.45\textwidth}
    \includegraphics[width=\linewidth]{images/gray_gpu.png}
    \caption{Grayscale (GPU)}
  \end{subfigure}
  \hfill
  \begin{subfigure}[b]{0.45\textwidth}
    \includegraphics[width=\linewidth]{images/runtime_plot.png}
    \caption{Runtime vs Block Size}
  \end{subfigure}
  \caption{GPU results and performance graph}
\end{figure}

\subsection{Measured Runtime}
\begin{center}
\begin{tabular}{|c|c|}
\hline
\textbf{Block Size} & \textbf{Runtime (s)} \\ \hline
4 & 0.00082 \\ 
8 & 0.00048 \\ 
16 & 0.00038 \\ 
32 & 0.00033 \\ 
64 & 0.00033 \\ 
128 & 0.00034 \\ 
256 & 0.00035 \\ 
512 & 0.00035 \\ 
1024 & 0.00036 \\ \hline
\end{tabular}
\end{center}

% ====== Section 3 ======
\section{Discussion and Conclusion}
GPU execution shows massive speedup compared to CPU due to parallel computation.  
Performance improved as block size increased up to 64–128 threads, then stabilized.  
Both CPU and GPU outputs were visually identical, confirming correctness.  


\end{document}
