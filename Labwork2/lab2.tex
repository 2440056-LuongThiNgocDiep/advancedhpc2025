\documentclass[hidelinks,12pt,a4paper]{article}
\usepackage{graphicx}
\usepackage{amsmath}
\usepackage{siunitx}
\usepackage{tikz}
\usetikzlibrary{automata, arrows, positioning}
\usepackage{hyperref}
\usepackage{cite}
\usepackage[T1]{fontenc}
\usepackage[utf8]{inputenc}
\usepackage{xcolor}
\usepackage{geometry}
\geometry{left=1.2in,top=1in,bottom=1in,right=1in}
\usepackage{titlesec}
\usepackage{fancyhdr}
\usepackage{float}
\definecolor{blue}{RGB}{31,56,100}

\titleformat{\section}
  {\normalfont\LARGE\bfseries}
  {\thesection}{1em}{}
\titleformat{\subsection}
  {\normalfont\Large\bfseries}
  {\thesubsection}{1em}{}
\titleformat{\subsubsection}
  {\normalfont\large\bfseries}
  {\thesubsubsection}{1em}{}

\begin{document}
\pagenumbering{gobble}

\begin{titlepage}
    \begin{tikzpicture}[remember picture,overlay,inner sep=0,outer sep=0]
        \draw[black!70!black,line width=1.5pt]
            ([xshift=-0.65in,yshift=-1cm]current page.north east) --
            ([xshift=0.65in,yshift=-1cm]current page.north west) --
            ([xshift=0.65in,yshift=1cm]current page.south west) --
            ([xshift=-0.65in,yshift=1cm]current page.south east) -- cycle;
    \end{tikzpicture}

    \begin{center}
        {\large\uppercase{University of Science and Technology of Hanoi}} \\[1.5cm]
        \includegraphics[width=0.7\linewidth]{images/usth.png} \\[1cm]
        {\huge \bfseries \uppercase{Labwork's Report}} \\[1cm]
        {\large \bfseries 2440056 - Luong Thi Ngoc} \\[0.5cm]
        {\huge \bfseries {Advanced Programming for HPC}} \\[1cm]
        \rule{\linewidth}{0.3mm} \\[0.4cm]
        {\Huge \bfseries \color{blue} Labwork 2: Get to know your GPU} \\
        \rule{\linewidth}{0.3mm} \\[0.7cm]
        \large Academic Year: 2024--2026
    \end{center}
\end{titlepage}

\clearpage
\pagenumbering{arabic}

\section{Introduction}
In this labwork, I explored the GPU device available in Google Colab using the \texttt{numba.cuda} library. 

\subsection{Device name}
The GPU used in this lab is:

\begin{figure}[H]
    \centering
    \includegraphics[width=0.75\linewidth]{fig/ID and Name of the GPUs.png}
    \caption{ID and Name of the GPUs}
    \label{fig:placeholder}
\end{figure}
\subsection{Core info: multiprocessor count, core count}
\begin{figure}[H]
    \centering
    \includegraphics[width=0.75\linewidth]{fig/multiprocessor count and core count.png}
    \caption{Multiprocessor count, core count}
    \label{fig:placeholder}
\end{figure}
\begin{itemize}
    \item Multiprocessor count (SMs): 40
    \item CUDA cores per SM: 64
    \item {Total CUDA cores:} 2560
\end{itemize}

To determine the total number of CUDA cores, I used the mapping between compute capability and cores per multiprocessor, as referenced in StackOverflow \cite{stackoverflow-cuda-cores}.

\subsection{Memory info}
\begin{figure}[H]
    \centering
    \includegraphics[width=0.8\linewidth]{fig/Memory info.png}
    \caption{Memory info}
    \label{fig:placeholder}
\end{figure}

\section{Conclusion}
The Tesla T4 GPU available in Colab provides 2560 CUDA cores and 14.74~GB of global memory.


\begin{thebibliography}{9}
\bibitem{stackoverflow-cuda-cores} 
StackOverflow, \emph{How can I get the number of CUDA cores in my GPU using Python and Numba?}  
\url{https://stackoverflow.com/questions/63823395/how-can-i-get-the-number-of-cuda-cores-in-my-gpu-using-python-and-numba}
\end{thebibliography}

\end{document}
